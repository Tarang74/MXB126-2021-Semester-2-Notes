\documentclass{article}
\usepackage{template}

\usepackage{chngcntr} % Reset counter within sections
\usepackage{multicol}

\counterwithin*{equation}{section}
\counterwithin*{equation}{subsection}

\pagenumbering{gobble}
\geometry{
	a4paper,
	margin = 10mm
}

\date{}
 
\begin{document}
\begin{multicols}{2}[When solving an initial value problem, always solve the general solution first.]
    \subsubsection*{First-Order ODEs}
    \noindent \textbf{Separable ODEs.} $\dv{y}{t} = f(y)g(t)$
    \begin{enumerate}[itemsep=1pt, parsep=1pt]
        \item Rewrite as: $f\dd{y} = g\dd{t}$.
        \item Integrate both sides: $\int f\dd{y} = \int g\dd{t} + C$.
        \item Rearrange for the explicit form of $y(t)$.
    \end{enumerate}
    \noindent \textbf{Linear ODEs.} $\dv{y}{t} + P(t)y = Q(t)$
    \begin{enumerate}[itemsep=1pt, parsep=1pt]
        \item Determine the integrating factor: $\mu(t)=\exp{\left( \int P \dd{t} \right)}$.
        \item Solve: $y(t)=\frac{1}{\mu}\left(\int Q \mu \dd{t} + C\right)$.
    \end{enumerate}
    \noindent \textbf{Linearisation.}
    \begin{gather*}
        f(x) \approx f(x_0) + f'(x_0)(x-x_0) \\
        f\bigl(y(x)\bigr) \approx f\bigl(y(x_0)\bigr) + f'\bigl(y(x)\bigr)\bigl(y(x)-y(x_0)\bigr)
    \end{gather*}
    \subsubsection*{Constant Coefficient Linear ODEs}
    \noindent \textbf{Homogeneous ODEs.} $a \dv[2]{y}{t} + b \dv{y}{t} + c y = 0$
    \begin{enumerate}[itemsep=1pt, parsep=1pt]
        \item Substitute $y_h = \e^{rt}$ and solve characteristic equation: $ar^2 + br + c = 0$.
        \item Find homogeneous solution:
        \begin{description}
            \item \textbf{Real Distinct Roots ($r_1,\: r_2$).} \\
                $y_h(t) = c_1\e^{r_1t} + c_2\e^{r_2t}$.
            \item \textbf{Real Repeated Roots ($r$).} \\
                $y_h(t) = c_1\e^{rt} + c_2t\e^{rt}$
            \item \textbf{Complex Conjugate Roots ($r_{1,\: 2} = \alpha \pm \beta i$).} \\
                $y_h(t) = c_1\e^{\alpha t}\cos{\left( \beta t \right)} + c_2\e^{\alpha t}\sin{\left( \beta t \right)}$.
        \end{description}
    \end{enumerate} 
    \noindent \textbf{Nonhomogeneous ODEs.} $a \dv[2]{y}{t} + b \dv{y}{t} + c y = Q(t)$
    \begin{enumerate}[itemsep=1pt, parsep=1pt]
        \item Determine $y_h$.
        \item If $y_p$ is linearly independent to $y_h$, multiply $y_p$ by $t$.
        \item Substitute $y_p$ and solve for undetermined coefficients, using the table below.
        \begin{table}[H]
            \centering
            \begin{tabular}{c | c}
                \toprule
                $Q(t)$ & $y_p$ \\
                \midrule
                a constant & $A$ \\ 
                $n$th degree polynomial & $\sum_{i = 0}^n A_i t^i$ \\ 
                $\e^{\alpha t}$ & $A\e^{\alpha t}$ \\ 
                $\cos{\left( \alpha t \right)}$ or $\sin{\left( \alpha t \right)}$ & $A\cos{\left( \alpha t \right)} + B\sin{\left( \alpha t \right)}$ \\ 
                sum/product of above & sum/product of above \\ 
                \bottomrule
            \end{tabular}
            \vspace{1ex}

            {\raggedleft\textit{If $Q(t)$ contains multiple forms, simplify the coefficients before substituting $y_p$.}\par}
        \end{table}
        \item Find general solution: $y = y_h + y_p$.
    \end{enumerate}
\end{multicols}
\section{Systems of Ordinary Differential Equations}
A first-order system of differential equations has the form
\begin{equation*}
    \left\{
        \setlength\arraycolsep{0pt}
        \begin{array}{ c >{{}}c<{{}} c >{{}}c<{{}} c >{{}}c<{{}} c >{{}}c<{{}} c  }
        x'_1               &=& a_{11}x_1                         &+& a_{12}x_2                         &+& \cdots &+& a_{1n}x_n \\
        x'_2               &=& a_{21}x_1                         &+& a_{22}x_2                         &+& \cdots &+& a_{2n}x_n \\
        \vdotswithin{x'_3} & & \vdotswithin{a_{31}}\phantom{x_1} & & \vdotswithin{a_{32}}\phantom{x_2} & &        & & \vdotswithin{a_{3n}}\phantom{x_n} \\ 
        x'_n               &=& a_{n1}x_1                         &+& a_{n2}x_2                         &+& \cdots &+& a_{nn}x_n 
        \end{array}
    \right.
\end{equation*}
where $x_1=\func{x_1}{t},\: x_2=\func{x_2}{t},\: \dots,\: x_n=\func{x_n}{t}$ are the 
functions to be determined. In matrix form, the system can be written as
\begin{align*}
    \dv{t}\mqty[x_1 \\ x_2 \\ \vdots \\ x_n] &= \mqty[
        a_{11} & a_{12} & \cdots & a_{1n} \\
        a_{21} & a_{22} & \cdots & a_{2n} \\
        \vdots & \vdots &        & \vdots \\
        a_{n1} & a_{n2} & \cdots & a_{nn}
    ] \mqty[x_1 \\ x_2 \\ \vdots \\ x_n] \\
    \symbf{x}' &= \symbfit{A} \symbf{x}
\end{align*}
\paragraph{Higher-Order ODEs}
A higher-order linear differential equation can be solved by first converting it to a first-order linear 
system. Consider the $n$th-order homogeneous differential equation
\begin{equation*}
    y^{\left( n \right)} + a_1 y^{\left( n-1 \right)} + \cdots + a_{n-1} y' + a_n y = 0
\end{equation*}
Let
\begin{align*}
    x_1 &= y \\
    x_2 &= y' \\
    &\vdotswithin{=} \\
    x_n &= y^{\left( n-1 \right)}
\end{align*}
so that $\symbfit{x}=\mqty[x_1 & x_2 & \cdots & x_n]^\top$. Then the differential equation can
be expressed as the following first-order linear system of differential equations
\begin{equation*}
    \dv{t}\mqty[
        x_1 \\
        x_2 \\
        \vdotswithin{x_3} \\
        x_n	
    ] = \mqty[
        0 & 1 & 0 & \cdots & 0 \\
        0 & 0 & 1 & \cdots & 0 \\
        \vdots & \vdots & \vdots & \ddots & \vdots \\
        0 & 0 & 0 & \cdots & 1 \\
        -a_n & -a_{n-1} & -a_{n-2} & \cdots & -a_1
    ] \mqty[
        x_1 \\
        x_2 \\
        \vdotswithin{x_3} \\
        x_n	
    ]
\end{equation*}
\textbf{Solution Form}
Like the homogeneous case, we will guess a solution of the form
\begin{equation*}
    \symbf{x} = \symbf{q}\e^{\lambda t}
\end{equation*}
which allows for the following substitution
\begin{align*}
    \lambda \symbf{q}\e^{\lambda t} &= \symbfit{A} \symbf{q}\e^{\lambda t} \\
    \symbfit{A} \symbf{q}\e^{\lambda t} - \lambda \symbf{q}\e^{\lambda t} &= \symbf{0} \\
    \left( \symbfit{A} - \lambda\symbfit{I} \right) \symbf{q}\e^{\lambda t} &= \symbf{0} \\
    \left( \symbfit{A} - \lambda\symbfit{I} \right) \symbf{q} &= \symbf{0}
\end{align*}
This equation has the trivial solution $\symbf{q}=\symbf{0}$, however for a fundamental set of solutions,
we must let $\symbfit{A} - \lambda\symbfit{I}$ be singular.
\subsubsection{Characteristic Equation}
To determine the eigenvalues $\lambda$ of the matrix $\symbf{A}$, we must solve the characteristic equation
associated with the system of ODEs. Namely,
\begin{equation*}
    \det{\left( \symbfit{A} - \lambda\symbfit{I} \right)} = 0
\end{equation*}
These eigenvalues can then be used to solve the eigenvectors of $\symbfit{A}$
\textbf{Solving a System of ODEs}
\begin{enumerate}
    \item Model the system of ODEs in the form $\symbf{x}' = \symbfit{A} \symbf{x}$
    \item Solve the characteristic equation for the eigenvalues of $\symbfit{A}$
    \item Solve the corresponding eigenvectors of $\symbfit{A}$ by solving $\left( \symbfit{A} - \lambda\symbfit{I} \right) \symbf{q} = \symbf{0}$
    \item Write the general solution: $\symbf{x} = c_1\symbf{q}_1\e^{\lambda_1 t} + c_2\symbf{q}_2\e^{\lambda_2 t}$
    \item Apply initial conditions to solve $c_1$ and $c_2$
\end{enumerate}
\end{document}
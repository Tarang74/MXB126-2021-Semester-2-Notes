\documentclass{article}
\usepackage{LaTeX-Submodule/template}

\usepackage{chngcntr} % Reset counter within sections
\usepackage{multicol}
\usepackage{textcomp, upquote}

\counterwithin*{equation}{section}
\counterwithin*{equation}{subsection}

\pagestyle{fancy}
\setlength\headheight{24pt}
\setlength\parindent{0pt}

\lhead{\className}
\rhead{\leftmark}
\cfoot{\thepage}

\newcommand{\className}{Engineering Computation}
\newcommand{\classTime}{Semester 2, 2021}
\newcommand{\classInstructorName}{Dr Michael Dallaston}

\usepackage[
    type={CC},
    modifier={by-nc-sa},
    version={4.0},
    imagewidth={5em},
    hyphenation={raggedright}
]{doclicense}

\date{}

\begin{document}

\begin{titlepage}
    \vspace*{\fill}
    \begin{center}
        \LARGE{\textbf{\className}}
        \texorpdfstring{\\}{ }
        \texorpdfstring{\vspace{0.1in}}{ }
        \normalsize{\classTime}
        \texorpdfstring{\\}{ }
        \texorpdfstring{\vspace{0.1in}}{ }
        \normalsize\textit{\classInstructorName}
        \texorpdfstring{\\}{ }
        \texorpdfstring{\vspace{0.2in}}{ }
        \textsc{Tarang Janawalkar}
    \end{center}
    \vspace*{\fill}
    \doclicenseThis
    \thispagestyle{empty}
\end{titlepage}
\newpage

\tableofcontents
\newpage

\lstset{language=Matlab, upquote=true}
\lstset{morekeywords={randi, false, imshow, drawpolygon, polyarea, drawline, ones, imread, VideoWriter, XData, YData, getFrame, writeVideo, deg2rad, soundsc, resample, audiowrite, sind}}
\lstset{
    basicstyle = \ttfamily, columns = fullflexible, keepspaces = true
}

\section{MATLAB Functions}

\begin{table}[H]
    \centering
    \begin{tabular}{c | c}
        \toprule
        Function Syntax         & Function Output              \\
        \midrule
        \lstinline!y = sin(x)!  & Sine with $x$ in radians.    \\
        \lstinline!y = sind(x)! & Sine with $x$ in degrees.    \\
        \lstinline!y = asin(x)! & Arcsine with $y$ in radians. \\
        \lstinline!y = exp(x)!  & $\e^x$.                      \\
        \lstinline!y = log(x)!  & $\ln{\left( x \right)}$.     \\
        \bottomrule
    \end{tabular}
    \caption{Common mathematical functions in MATLAB.}
\end{table}
All the above functions are element-wise.
\begin{table}[H]
    \centering
    \begin{tabular}{c | c}
        \toprule
        Function Syntax                   & Function Output(s)                                               \\
        \midrule
        \lstinline!A = zeros(m, n)!       & Creates an $m \times n$ matrix containing zeros.                 \\
        \lstinline!A = ones(m, n)!        & Creates an $m \times n$ matrix containing ones.                  \\
        \lstinline!I = eye(m)!            & Creates an $m \times m$ identity matrix.                         \\
        \lstinline!a = linspace(a, b, x)! & Creates an evenly spaced vector with bounds $\left[a, b\right]$. \\
        \lstinline!y = length(A)!         & The largest dimension of $A$.                                    \\
        \lstinline![m, n] = size(A)!      & The dimensions of $A$.                                           \\
        \lstinline!y = min(a)!            & The minimum value in the vector $a$.                             \\
        \lstinline!y = max(a)!            & The maximum value in the vector $a$.                             \\
        \bottomrule
    \end{tabular}
    \caption{Matrices and arrays in MATLAB.}
\end{table}
When manipulating matrices, \verb!*!, \verb!^!, perform matrix operations, while prepending an operator with a dot (\verb!.!) performs an element-wise operation.
\subsection{Plotting}
\begin{table}[H]
    \centering
    \begin{tabular}{c | c}
        \toprule
        Function Syntax                    & Function Output(s)                                                    \\
        \midrule
        \lstinline!plot(x, y)!             & Plots given $x$ and $y$ coordinate vectors.                           \\
        \lstinline!fplot(@f, [a, b])!      & Plots the anonymous function over the domain $\left[ a,\: b \right]$. \\
        \lstinline!title('string')!        & Adds title to current plot.                                           \\
        \lstinline!xlabel('string')!       & Adds $x$-axis label to current plot.                                  \\
        \lstinline!ylabel('string')!       & Adds $y$-axis label to current plot.                                  \\
        \lstinline!legend('string1', ...)! & Adds legend to plot.                                                  \\
        \lstinline!figure!                 & Creates a new figure.                                                 \\
        \bottomrule
    \end{tabular}
    \caption{Plotting in MATLAB.}
    % \label{}
\end{table}
\section{Operations in MATLAB}
\subsection{Conditional Operations}
\begin{multicols}{2}
    \begin{lstlisting}
if expression
    statements
else if expression
    statements
else
    statements
end
    \end{lstlisting}
    \columnbreak
    Code inside an \lstinline{if} statement only executes if the expression is true. Note that only one branch will execute depending on which expression is true.
\end{multicols}
\subsection{Iterative Operations}
\begin{multicols}{2}
    \begin{lstlisting}
while expression
    statements
end
    \end{lstlisting}
    \columnbreak
    Statements inside a \lstinline{while} loop repeat while the expression is true.
\end{multicols}
\begin{multicols}{2}
    \begin{lstlisting}
for index = values
    statements
end
    \end{lstlisting}
    \columnbreak
    Statements inside a \lstinline{for} loop execute a specific number of times, based on the length of \lstinline{values}.
\end{multicols}
\section{Differential Equations}
\begin{definition}
    A differential equation is an equation that involves the derivatives of a function as well as the function itself.
    An ordinary differential equation (ODE) is a differential equation of a function with only one independent variable.
\end{definition}
\subsection{Electrical Systems}
\begin{theorem}[VI Relationship between Resistors]
    \begin{equation*}
        v = i R
    \end{equation*}
\end{theorem}
\begin{theorem}[VI Relationship between Inductors]
    \begin{equation*}
        v = L \dv{i}{t}
    \end{equation*}
\end{theorem}
\begin{theorem}[VI Relationship between Capacitors]
    \begin{equation*}
        i = C \dv{v}{t}
    \end{equation*}
\end{theorem}
\begin{theorem}[Kirchhoff's Voltage Law]
    The sum of all voltages around a loop equals zero.
    \begin{equation*}
        \sum v_{\mathrm{loop}} = 0
    \end{equation*}
\end{theorem}
\begin{theorem}[Kirchhoff's Current Law]
    The sum of all currents into a node equals zero.
    \begin{equation*}
        \sum i_{\mathrm{node}} = 0
    \end{equation*}
\end{theorem}
\subsection{Mechanical Systems}
\begin{theorem}[Newton's Second Law]
    \begin{equation*}
        \sum F = \dv{p}{t}
    \end{equation*}
    where $p = mv$.
\end{theorem}
\begin{theorem}[Thrust Force]
    \begin{equation*}
        F_T = (c - v) d_f
    \end{equation*}
\end{theorem}
\begin{theorem}[Force of Gravity]
    \begin{equation*}
        F_g = mg
    \end{equation*}
\end{theorem}
\begin{theorem}[Force of a Spring]
    \begin{equation*}
        F = -kx
    \end{equation*}
\end{theorem}
\section{First-Order Ordinary Differential Equations}
\subsection{Separable ODEs}
\begin{equation*}
    \dv{y}{t} = F(y,\: t)
\end{equation*}
\begin{enumerate}
    \item Rewrite the equation in the form: $f(y)\dd{y} = g(t)\dd{t}$.
    \item Integrate both sides: $\int f(y)\dd{y} = \int g(t)\dd{t}$.
    \item Rearrange for the explicit form of $y(t)$.
\end{enumerate}
\subsection{Linear ODEs}
Let $P=P(t)$, $Q=Q(t)$ and $\mu = \mu(t)$
\begin{equation*}
    \dv{y}{t} + Py = Q
\end{equation*}
\begin{enumerate}
    \item Determine the integrating factor: $\displaystyle \mu=\exp{\left( \int P \dd{t} \right)}$.
    \item Solve:
          \begin{equation*}
              y=\frac{1}{\mu}\left(\int Q \mu \dd{t} + C\right)
          \end{equation*}
\end{enumerate}
\begin{proof}
    To solve a first-order linear differential equation, determine an integrating factor $\mu = \mu(t)$ such that
    \begin{equation}
        P \mu = \dv{\mu}{t} \label{eq:integrating_factor}
    \end{equation}
    Multiplying the equation by $\mu$ gives
    \begin{align*}
        \mu \dv{y}{t} + P \mu y             & = Q \mu                                           \\
        \mu \dv{y}{t} + \dv{\mu}{t} y       & = Q \mu                                           \\
        \dv{t}\bigl(\mu y\bigr)             & = Q \mu                                           \\
        \int \dv{t}\bigl(\mu y\bigr) \dd{t} & = \int Q \mu \dd{t}                               \\
        \mu y                               & = \int Q \mu \dd{t}                               \\
        y                                   & = \frac{1}{\mu}\left(\int Q \mu \dd{t} + C\right)
    \end{align*}
    To determine $\mu$ we can rearrange \hyperref[eq:integrating_factor]{Equation \ref{eq:integrating_factor}} into
    \begin{align*}
        P = \frac{1}{\mu}\dv{\mu}{t}
    \end{align*}
    By recognition, this is the derivative of the natural logarithm of $\mu$ with respect to $t$.
    \begin{align*}
        P             & = \dv{t}(\ln{\left( \mu \right)})                       \\
        \int P \dd{t} & = \int \dv{t}\bigl(\ln{\left( \mu \right)}\bigr) \dd{t} \\
        \int P \dd{t} & = \ln{\left( \mu \right)}                               \\
        \mu           & = \exp{\left( \int P \dd{t} \right)}
    \end{align*}
\end{proof}
\subsection{Solution using Linearisation}
A function can be linearised by using its 1st degree Taylor polynomial near $a$.
\begin{equation*}
    f(x) \approx f(a) + f'(a)(x-a) + \mathcal{O}(x^2)
\end{equation*}
This new polynomial can be substituted to form a linear ODE, which can be solved using an integrating factor.
\newpage
\section{Second-Order Ordinary Differential Equations}
\subsection{Constant Coefficient Linear ODEs}
\begin{equation*}
    a \dv[2]{y}{t} + b \dv{y}{t} + c y = Q(t)
\end{equation*}
where $a,\:b,\:c$ are constants.
\subsection{Linearity of Solutions}
\begin{theorem}[Principle of Superposition]\label{theorem:superposition}
    As the given ODE is linear, if $y_1(t)$ is a solution to the equation
    \begin{equation*}
        a \dv[2]{y_1}{t} + b \dv{y_1}{t} + cy_1 = Q_1(t)
    \end{equation*}
    and $y_2(t)$ is a solution to
    \begin{equation*}
        a \dv[2]{y_2}{t} + b \dv{y_2}{t} + cy_2 = Q_2(t)
    \end{equation*}
    then for the function $y = c_1y_1+c_2y_2$
    \begin{equation*}
        a \dv[2]{y}{t} + b \dv{y}{t} + cy = c_1Q_1(t) + c_2Q_2(t)
    \end{equation*}
    where $c_1$ and $c_2$ are constants.
\end{theorem}
% \begin{proof}
% \begin{align*}
%     a \dv[2]{y}{t} + b \dv{y}{t} + cy &= a \dv[2]{t}(k_1y_1+k_2y_2) + b \dv{t}(k_1y_1+k_2y_2) + c(k_1y_1+k_2y_2) \\
%     &= k_1\left( a \dv[2]{y}{t} + b \dv{y}{t} + cy \right)a \dv[2]{t}(k_1y_1+k_2y_2) + b \dv{t}(k_1y_1+k_2y_2) + c(k_1y_1+k_2y_2) \\
% \end{align*}
% \end{proof}
\subsection{Homogeneous ODEs}
\begin{definition}
    A homogeneous ODE has $Q(t)=0$, which gives
    \begin{equation*}
        a \dv[2]{y}{t} + b \dv{y}{t} + c y = 0
    \end{equation*}
\end{definition}
This differential equation has a solution of the form:
\begin{equation*}
    y_H = \e^{rt}
\end{equation*}
\subsection{Characteristic Equation}
By making the substitution $y=\e^{rt}$, we get
\begin{align*}
    a \dv[2]{y_H}{t} + b \dv{y_H}{t} + c y_H & = 0 \\
    ar^2\e^{rt} + br\e^{rt} + c\e^{rt}       & = 0 \\
    (ar^2 + br + c)\e^{rt}                   & = 0 \\
    ar^2 + br + c                            & = 0
\end{align*}
This is known as the characteristic or \textit{auxiliary} equation. The next step is to calculate the roots of the equation.
\begin{description}
    \item[Real Distinct Roots.] If $b^2 > 4ac$.
    \item[Real Repeated Roots.] If $b^2 = 4ac$.
    \item[Complex Conjugate Roots.] If $b^2 < 4ac$.
\end{description}
\subsubsection{Real Distinct Roots}
Given $r_1$ and $r_2$ are real and distinct:
\begin{align*}
    y_1(t) = \e^{r_1t} &  & y_2(t) = \e^{r_2t}
\end{align*}
Hence the solution to the homogeneous equation is given by:
\begin{equation*}
    y_H(t) = c_1\e^{r_1t} + c_2\e^{r_2t}
\end{equation*}
\subsubsection{Real Repeated Roots}
Given $r$ is a repeated root:
\begin{align*}
    y_1(t) = \e^{rt} &  & y_2(t) = t\e^{rt}
\end{align*}
Hence the solution to the homogeneous equation is given by:
\begin{equation*}
    y_H(t) = c_1\e^{rt} + c_2t\e^{rt}
\end{equation*}
\subsubsection{Complex Conjugate Roots}
Given $r_1 = \alpha + \beta i$ and $r_2 = \alpha - \beta i$ are complex conjugates:
\begin{align*}
    y_1(t) = \e^{r_1t} &  & y_2(t) = \e^{r_2t}
\end{align*}
Hence the solution to the homogeneous equation is given by:
\begin{equation*}
    y_H(t) = c_1\e^{\alpha t}\cos{\left( \beta t \right)} + c_2\e^{\alpha t}\sin{\left( \beta t \right)}
\end{equation*}
\subsection{Nonhomogeneous ODE}
A nonhomogeneous differential equation is of the form
\begin{equation*}
    a \dv[2]{y}{t} + b \dv{y}{t} + c y = Q(t)
\end{equation*}
where $Q(t)\neq 0$.
\subsection{General Solution of a Nonhomogeneous ODE}
Recall that the solutions to any linear ODE are additive, so that if a solution $y_P$ satisfies
the nonhomogeneous ODE, and $y_H$ satisfies the homogeneous ODE,
\begin{equation*}
    y = y_H + y_P
\end{equation*}
must also satisfy the ODE.
\subsection{Undetermined Coefficients}
To solve for $y_P$, we substitute a guess, and
determined the coefficients from the ODE itself.

The particular solution will depend on what $Q(t)$ looks like.
\begin{table}[H]
    \centering
    \begin{tabular}{c | c}
        \toprule
        $Q(t)$                                                          & $y_P$                                                             \\
        \midrule
        a constant                                                      & $A$                                                               \\
        $n$th degree polynomial                                         & $A_0 + A_1t + \cdots + A_{n-1}t^{n-1} + A_nt^n$                   \\
        $\e^{\alpha t}$                                                 & $A\e^{\alpha t}$                                                  \\
        $\cos{\left( \alpha t \right)}$                                 & $A\cos{\left( \alpha t \right)} + B\sin{\left( \alpha t \right)}$ \\
        $\sin{\left( \alpha t \right)}$                                 & $A\cos{\left( \alpha t \right)} + B\sin{\left( \alpha t \right)}$ \\
        $\cos{\left( \alpha t \right)} + \sin{\left( \alpha t \right)}$ & $A\cos{\left( \alpha t \right)} + B\sin{\left( \alpha t \right)}$ \\
        \bottomrule
    \end{tabular}
    \caption{Particular solutions for undetermined coefficients.}
    % \label{}
\end{table}
\subsection{Special Forms}
\subsubsection{Product of Forms}
If $Q(t)$ is a product of the functions shown above, then the
particular solutions are also multiplied together and any coefficients are simplified.
\subsubsection{Sum of Forms}
If $Q(t)$ is a sum of the functions shown above, then the
particular solutions are also added together.
\subsubsection{Linearly Dependent Case}
If $Q(t)$ is similar to any homogenous solution, then by \textit{definition} of
a homogeneous solution, the solution will be $0$. Hence, $y_P$ must be multiplied by $t$
to ensure that the particular solution is linearly independent to the homogeneous solutions,
in order to form a \textit{fundamental set of solutions}.
\subsection{Solving the Particular Solution}
\begin{enumerate}
    \item Solve $y_H$
    \item Find an appropriate form for $y_P$
    \item Ensure that $y_P$ is linearly independent to the homogeneous solutions
    \item Substitute $y_P$ into the nonhomogeneous ODE and solve for the undetermined coefficients
    \item Find the general solution $y = y_H + y_P$
    \item Apply initial conditions
\end{enumerate}
\section{Systems of Ordinary Differential Equations}
A first-order system of differential equations has the form
\begin{equation*}
    \left\{
    \setlength\arraycolsep{0pt}
    \begin{array}{ c >{{}}c<{{}} c >{{}}c<{{}} c >{{}}c<{{}} c >{{}}c<{{}} c  }
        x'_1               & = & a_{11}x_1                         & + & a_{12}x_2                         & + & \cdots & + & a_{1n}x_n                         \\
        x'_2               & = & a_{21}x_1                         & + & a_{22}x_2                         & + & \cdots & + & a_{2n}x_n                         \\
        \vdotswithin{x'_3} &   & \vdotswithin{a_{31}}\phantom{x_1} &   & \vdotswithin{a_{32}}\phantom{x_2} &   &        &   & \vdotswithin{a_{3n}}\phantom{x_n} \\
        x'_n               & = & a_{n1}x_1                         & + & a_{n2}x_2                         & + & \cdots & + & a_{nn}x_n
    \end{array}
    \right.
\end{equation*}
where $x_1=x_1(t),\: x_2=x_2(t),\: \dots,\: x_n=x_n(t)$ are the
functions to be determined. In matrix form, the system can be written as
\begin{align*}
    \dv{t}\mqty[x_1                                        \\ x_2 \\ \vdots \\ x_n] &= \mqty[
    a_{11}     & a_{12}                  & \cdots & a_{1n} \\
    a_{21}     & a_{22}                  & \cdots & a_{2n} \\
    \vdots     & \vdots                  &        & \vdots \\
    a_{n1}     & a_{n2}                  & \cdots & a_{nn}
    ] \mqty[x_1                                            \\ x_2 \\ \vdots \\ x_n] \\
    \symbf{x}' & = \symbfit{A} \symbf{x}
\end{align*}
\subsection{Higher-Order ODEs}
A higher-order linear differential equation can be solved by first converting it to a first-order linear
system. Consider the $n$th-order homogeneous differential equation
\begin{equation*}
    y^{\left( n \right)} + a_1 y^{\left( n-1 \right)} + \cdots + a_{n-1} y' + a_n y = 0
\end{equation*}
Let
\begin{align*}
    x_1 & = y                      \\
    x_2 & = y'                     \\
        & \vdotswithin{=}          \\
    x_n & = y^{\left( n-1 \right)}
\end{align*}
so that $\symbfit{x}=\mqty[x_1 & x_2 & \cdots & x_n]^\top$. Then the differential equation can
be expressed as the following first-order linear system of differential equations
\begin{equation*}
    \dv{t}\mqty[
        x_1 \\
        x_2 \\
        \vdotswithin{x_3} \\
        x_n
    ] = \mqty[
    0 & 1 & 0 & \cdots & 0 \\
    0 & 0 & 1 & \cdots & 0 \\
    \vdots & \vdots & \vdots & \ddots & \vdots \\
    0 & 0 & 0 & \cdots & 1 \\
    -a_n & -a_{n-1} & -a_{n-2} & \cdots & -a_1
    ] \mqty[
        x_1 \\
        x_2 \\
        \vdotswithin{x_3} \\
        x_n
    ]
\end{equation*}
\subsection{Solution Form}
Like the homogeneous case, we will guess a solution of the form
\begin{equation*}
    \symbf{x} = \symbf{q}\e^{\lambda t}
\end{equation*}
which allows for the following substitution
\begin{align*}
    \lambda \symbf{q}\e^{\lambda t}                                         & = \symbfit{A} \symbf{q}\e^{\lambda t} \\
    \symbfit{A} \symbf{q}\e^{\lambda t} - \lambda \symbf{q}\e^{\lambda t}   & = \symbf{0}                           \\
    \left( \symbfit{A} - \lambda\symbfit{I} \right) \symbf{q}\e^{\lambda t} & = \symbf{0}                           \\
    \left( \symbfit{A} - \lambda\symbfit{I} \right) \symbf{q}               & = \symbf{0}
\end{align*}
This equation has the trivial solution $\symbf{q}=\symbf{0}$, however for a fundamental set of solutions,
we must let $\symbfit{A} - \lambda\symbfit{I}$ be singular.
\subsubsection{Characteristic Equation}
To determine the eigenvalues $\lambda$ of the matrix $\symbf{A}$, we must solve the characteristic equation
associated with the system of ODEs. Namely,
\begin{equation*}
    \det{\left( \symbfit{A} - \lambda\symbfit{I} \right)} = 0
\end{equation*}
These eigenvalues can then be used to solve the eigenvectors of $\symbfit{A}$
\subsection{Solving a System of ODEs}
\begin{enumerate}
    \item Model the system of ODEs in the form $\symbf{x}' = \symbfit{A} \symbf{x}$
    \item Solve the characteristic equation for the eigenvalues of $\symbfit{A}$
    \item Solve the corresponding eigenvectors of $\symbfit{A}$ by solving $\left( \symbfit{A} - \lambda\symbfit{I} \right) \symbf{q} = \symbf{0}$
    \item Write the general solution: $\symbf{x} = c_1\symbf{q}_1\e^{\lambda_1 t} + c_2\symbf{q}_2\e^{\lambda_2 t}$
    \item Apply initial conditions to solve $c_1$ and $c_2$
\end{enumerate}
\section{Probability}
\begin{definition}[Random Variables]
    A random variable $X$ is a measurable variable that doesn't hold a definitive value.
\end{definition}
\begin{definition}[Discrete Random Variables]
    A discrete random variable $X$ has a countable number of possible values.
\end{definition}
\begin{definition}[Continuous Random Variables]
    A continuous random variable $X$ can take all values in a given interval.
\end{definition}
\begin{definition}[Probability]
    Probability is used to mean the chance that a particular event
    (or set of events) will occur, expressed on a linear scale from 0 to 1.
    The probability of the random variable $X$ taking the value $x$ is denoted
    \begin{equation*}
        \Pr{\left( X = x \right)}
    \end{equation*}
\end{definition}
\begin{definition}[Sample Space]
    Let the set of all possible outcomes of a random variables be called the sample
    space, denoted $\Omega$, of that random variable.

    Let $X$ be a random variable that can take on values $x\in\Omega$. Then
    for all $x\in\Omega$ there is an associated probability $p(x)$, such that
    \begin{equation*}
        \forall x\in\Omega : 0 < p(x) \leq 1
    \end{equation*}
    \begin{equation*}
        \sum p(x) = 1.
    \end{equation*}
\end{definition}
\subsection{Events}
\begin{definition}[Events]
    An event $A$ is a set of individual outcomes within $\Omega$.
    Then for some event $A\subset\Omega$, the probability is given by
    \begin{equation*}
        \Pr{\left( A \right)} = \sum_{x\in A} p(x).
    \end{equation*}
    The complementary event, denoted $A^C$ (also $\overline{A}$) is the set of all
    outcomes within the sample space that are not within $A$.
    \begin{equation*}
        \Pr{\left( A^C \right)} = 1 - \Pr{\left( A \right)}.
    \end{equation*}
\end{definition}
\begin{definition}[Combination of Events]
    Events can be combined with the two logical connectors AND and OR, which are
    equivalent to the intersection ($\cap$) and union ($\cup$) of set.
\end{definition}
\begin{definition}[Mutually Exclusive Events]
    If two events have no possible outcomes in common, they are mutually exclusive or disjoint events.
    \begin{equation*}
        A \cap B = \varnothing.
    \end{equation*}
    It follows that
    \begin{equation*}
        \Pr{\left( A \cap B \right)} = 0.
    \end{equation*}
\end{definition}
\begin{theorem}[Probability of Union]
    \begin{align*}
        \Pr{\left( A \cup B \right)} & = \Pr{\left( A \right)} + \Pr{\left( B \right)} - \Pr{\left( A \cap B \right)} \\
                                     & = 1 - \Pr{\left( A^C \cap B^C \right)}
    \end{align*}
\end{theorem}
\begin{definition}[Independent Events]
    Two events are independent if the outcome of one event has no influence on the outcome of the other.
    For these cases, the joint probability is given by
    \begin{equation*}
        \Pr{\left( A \cap B \right)} = \Pr{\left( A \right)} \Pr{\left( B \right)}.
    \end{equation*}
\end{definition}
\subsection{Dependent Events}
Two events are dependent if the outcome of one event influences the outcome of the other.
\begin{definition}[Conditional Probability]
    In the case of dependent events, we must use \linebreak conditional probability concepts in calculating
    joint probabilities.
    The probability of $A$ given that the event $B$ has occurred is
    \begin{equation*}
        \Pr{\left( A \;\middle|\; B \right)} = \frac{\Pr{\left( A \cap B \right)}}{\Pr{\left( B \right)}}
    \end{equation*}
\end{definition}
\begin{theorem}[Total Probability]
    If $B$ is a sample space of disjoint events $B_1,\: B_2,\: \dots,\: B_n$, then
    $A \cap B_1,\: A \cap B_2,\: \dots,\: A \cap B_N$ are also disjoint, and
    \begin{equation*}
        A = \left( A \cap B_1 \right) \cup \left( A \cap B_2 \right) \cup \cdots \cup \left( A \cap B_n \right)
    \end{equation*}
    This gives
    \begin{align*}
        \Pr{\left( A \right)} & = \sum_{i=1}^n \Pr{\left( A \cap B_i \right)}                                 \\
                              & = \sum_{i=1}^n \Pr{\left( A \;\middle|\; B_i \right)} \Pr{\left( B_i \right)}
    \end{align*}
\end{theorem}
\begin{theorem}[Bayes' Theorem]
    Using the commutativity of intersections, the rule for \linebreak conditional probability gives
    \begin{equation*}
        \Pr{\left( A \cap B \right)} = \Pr{\left( A \;\middle|\; B \right)} \Pr{\left( B \right)} = \Pr{\left( B \;\middle|\; A \right)} \Pr{\left( A \right)}
    \end{equation*}
    Therefore
    \begin{equation*}
        \Pr{\left( B \;\middle|\; A \right)} = \frac{\Pr{\left( A \;\middle|\; B \right)} \Pr{\left( B \right)}}{\Pr{\left( A \right)}}
    \end{equation*}
\end{theorem}
\section{Probability Distributions}
\begin{definition}[Discrete Random Variables]
    A discrete random variable has countably many outcomes.
\end{definition}
\begin{definition}[Continuous Random Variables]
    A continuous random variable can take an infinite number of individual outcomes.
\end{definition}
\begin{definition}[Probability Distributions]
    The probabilities of random variables make up a probability distribution.

    For discrete random variables, the distribution is described with a Probability
    Mass Function (PMF)
    \begin{equation*}
        p(x) = \Pr{\left( X = x \right)}
    \end{equation*}
    For continuous variables, the distribution is described with a Probability
    Density Function (PDF) and the associated Cumulative Distribution Function (CDF).

    Here, probabilities are represented by areas under the PDF:
    \begin{equation*}
        \Pr{\left( x_1 \leq X \leq x_2 \right)} = \int_{x_1}^{x_2} f(u) \dd{u}
    \end{equation*}
    and the CDF is defined as
    \begin{equation*}
        F(x) = \Pr{\left( X \leq x \right)} = \int_{-\infty}^{x} f(u) \dd{u}.
    \end{equation*}
    Note that $f(x)$ is a valid PDF provided
    \begin{equation*}
        f(x) \geq 0:\forall x \quad \text{and} \quad \int_{-\infty}^{\infty} f(u) \dd{u} = 1
    \end{equation*}
    while $F(x)$ is a valid CDF if:
    \begin{enumerate}
        \item $F$ is a non-decreasing right continuous function
        \item $\lim_{x\to-\infty} F(x) = 0$ and $\lim_{x\to\infty} F(x) = 1$
    \end{enumerate}
\end{definition}
\subsection{Summary Statistics}
\begin{definition}[Expectation]
    The expected value $\E{\left( X \right)}$, of a random variable is the average
    outcome that could be expected from an infinite number of observations of that
    variable. This is also known as the mean of the variable, denoted $\mu$.
    \begin{equation*}
        \E{\left( X \right)} =
        \begin{cases}
            \sum_{\Omega} x p(x)        & \text{for discrete variables}   \\
            \int_{\Omega} x f(x) \dd{x} & \text{for continuous variables}
        \end{cases}
    \end{equation*}
\end{definition}
\begin{definition}[Variance]
    The variance $\Var{\left( X \right)}$, of a random variable is a measure of spread
    of the distribution (defined as the average squared distance of each value from the mean).
    $\Var{\left( X \right)}$ is also denoted as $\sigma^2$.
    \begin{align*}
        \Var{\left( X \right)} & =
        \begin{cases}
            \sum_{\Omega} \left( x - \mu \right)^2 p(x)             & \text{for discrete variables}   \\
            \int_{\Omega} \left( x - \mu \right)^2 p(x) f(x) \dd{x} & \text{for continuous variables}
        \end{cases} \\
                               & = \E{\left( X^2 \right)} - \E{\left( X \right)}^2
    \end{align*}
\end{definition}
\begin{definition}[Standard Deviation]
    The standard deviation is defined as
    \begin{equation*}
        \sigma = \sqrt{\Var{\left( X \right)}}
    \end{equation*}
\end{definition}
\subsubsection{General Linear Combinations}
For a simple linear function of a random variable
\begin{align*}
    \E{\left( aX \pm b \right)}   & = a\E{\left( X \right)} \pm b \\
    \Var{\left( aX \pm b \right)} & = a^2\Var{\left( X \right)}
\end{align*}
For a general linear combination of two random variables,
\begin{align*}
    \E{\left( aX \pm bY \right)}   & = a\E{\left( X \right)} \pm b\E{\left( Y \right)}                                           \\
    \Var{\left( aX \pm bY \right)} & = a^2\Var{\left( X \right)} + b^2\Var{\left( Y \right)} \pm 2ab \Cov{\left( X,\: Y \right)}
\end{align*}
\begin{definition}[Covariance]
    Covariance is the joint variability of two random variables.
    \begin{align*}
        \Cov{\left( X,\: Y \right)} & = \E{\left( XY \right)} - \E{\left( X \right)} \E{\left( Y \right)}                 \\
                                    & = \Corr{\left( X,\: Y \right)} \sqrt{\Var{\left( X \right)} \Var{\left( Y \right)}}
    \end{align*}
\end{definition}
\begin{definition}[Correlation]
    The correlation of two random variables is any statistical \linebreak relationship between
    those two variables. The correlation $\Corr{\left( X,\: Y \right)}$ is usually denoted $\rho_{XY}$ or $\rho$,
    and it always satisfies $-1 \leq\rho\leq 1$.
\end{definition}
\section{Common Probability Distributions}
\subsection{Bernoulli Random Variables}
A Bernoulli random variable takes values according to the probability distribution:
\begin{equation*}
    X =
    \begin{cases}
        1 & \text{with probability $p\in\left[ 0,\: 1 \right]$} \\
        0 & \text{with probability $1 - p$}
    \end{cases}
\end{equation*}
This distribution has the following mean and variance
\begin{align*}
    \E{\left( X \right)}   & = p                     \\
    \Var{\left( X \right)} & = p\left( 1 - p \right)
\end{align*}
\subsection{Binomial Distribution}
The Binomial distribution describes a discrete random variable that is the sum of the outcomes of $n$
independent and identically distributed Bernoulli random variables.
The PMF for this distribution is given by
\begin{equation*}
    p(x) = \dbinom{n}{x} p^x \left( 1 - p \right)^{n - x}
\end{equation*}
where
\begin{equation*}
    \dbinom{n}{x} = \frac{n!}{x!\left( n - x \right)!}
\end{equation*}
This distribution has the following mean and variance
\begin{align*}
    \E{\left( X \right)}   & = np                     \\
    \Var{\left( X \right)} & = np\left( 1 - p \right)
\end{align*}
\subsection{Poisson Distribution}
The Poisson distribution describes discrete events that occur randomly at an average rate $\lambda$.
The PMF for this distribution is given by
\begin{equation*}
    p(x) = \frac{\mu^x\e^{-\mu}}{x!}
\end{equation*}
where $\mu = \lambda t$ is the average number of events occurring in any interval of length $t$.
This distribution has the following mean and variance
\begin{align*}
    \E{\left( X \right)}   & = \mu \\
    \Var{\left( X \right)} & = \mu
\end{align*}
\subsection{Uniform Distribution}
The uniform distribution describes a continuous random variable which is only non-zero over a finite interval $\left( a,\: b \right)$,
with every value within this interval being equally likely to occur. The PDF and CDF for this distribution are given by
\begin{align*}
    f(x) & = \frac{1}{b - a}     \\
    F(x) & = \frac{x - a}{b - a}
\end{align*}
This distribution has the following mean and variance
\begin{align*}
    \E{\left( X \right)}   & = \frac{a + b}{2}                   \\
    \Var{\left( X \right)} & = \frac{\left( b - a \right)^2}{12}
\end{align*}
\subsection{Exponential Distribution}
The exponential distribution describes the time between successive events that are occurring under the same conditions as Poisson
distributions. The PDF and CDF for this distribution are given by
\begin{align*}
    f(x) & = \lambda \exp{\left( -\lambda x \right)} \\
    F(x) & = 1 - \exp{\left( -\lambda x \right)}
\end{align*}
This distribution has the following mean and variance
\begin{align*}
    \E{\left( X \right)}   & = \frac{1}{\lambda}   \\
    \Var{\left( X \right)} & = \frac{1}{\lambda^2}
\end{align*}
\subsection{Normal Distribution}
\section{Statistical Inference}
The aim of statistical inference is to infer the true mean of a numerical variable from sample data.

In order to carry out inference, we must use data that is properly representative of the situation we wish to draw conclusions about.
\subsection{Sample Statistics}
Let $x$ be a set of sample data, with a sample size $n$.
\begin{theorem}[Sample Mean]
    The sample mean $\overline{X}$ is given by
    \begin{equation*}
        \overline{X} = \frac{1}{n} \sum_{i = 1}^{n} x_i
    \end{equation*}
\end{theorem}
\begin{theorem}[Sample Variance]
    The sample variance $s^2$ is given by
    \begin{align*}
        s^2 & = \frac{1}{n - 1} \sum_{i = 1}^{n} \left( x_i - \overline{X} \right)^2     \\
            & = \frac{1}{n - 1} \left( \sum_{i = 1}^{n} x_i^2 - n \overline{X}^2 \right)
    \end{align*}
\end{theorem}
\subsection{Sampling Distribution of \texorpdfstring{$\overline{X}$}{X bar}}
The Central Limit theorem states that the distribution of $\overline{X}$ will approach a
Normal distribution as $n \to \infty$. In most cases, $n = 30$ is sufficiently large for
a Normal distribution to accurately reflect the distribution of $\overline{X}$
\begin{theorem}[Expectation of a Sample Mean]
    \begin{equation*}
        \E{\left( \overline{X} \right)} = \mu
    \end{equation*}
\end{theorem}
\begin{theorem}[Variance of a Sample Mean]
    \begin{equation*}
        \Var{\left( \overline{X} \right)} = \frac{\sigma^2}{n}
    \end{equation*}
\end{theorem}
By standardising $\overline{X}$ we get
\begin{equation*}
    Z = \frac{\overline{X} - \mu}{\sigma/\sqrt{n}} \sim N(0,\: 1)
\end{equation*}
as $n\to\infty$. But as $\sigma$ is often unknown, and can vary between samples, the Normal
distribution may no long apply.
\subsection{Student's \texorpdfstring{$t$}{t}-distribution}
\begin{definition}
    When using sampled data, we can use the $t$-distribution for $\overline{X}$.
    \begin{equation*}
        T = \frac{\overline{X} - \mu}{s/\sqrt{n}} \sim t_{n-1}
    \end{equation*}
    where $t_{n-1}$ is Student's $t$-distribution with $n-1$ degrees of freedom.

    The $t$-distribution converges to the standard Normal distribution as $n\to\infty$.
\end{definition}
Values along the horizontal axis of the $t$-distribution can be denoted by $t_{n-1,\: 1-\alpha/2}$.
\begin{description}
    \item[$n-1$] represents the degrees of freedom.
    \item[$1-\alpha/2$] represents the CDF probability:
        \begin{equation*}
            \Pr{\left( T \leq t_{n-1,\: 1-\alpha/2} \right)} = 1-\alpha/2
        \end{equation*}
\end{description}
\subsection{Confidence Intervals}
\begin{definition}[Confidence Intervals]
    As $\overline{X}$ varies between samples, and is never equal to $\mu$, we can determine
    a range of values (based on the observation of $\overline{X}$) that is likely to contain
    the true mean. This range is called an interval estimate or confidence interval for $\mu$.

    The $\left( 1 - \alpha \right)$ confidence interval for $\mu$ is
    \begin{equation*}
        {CI}_{\left( 1 - \alpha \right)} = \overline{X} \pm t_{n-1,\: 1-\alpha/2} \frac{s}{\sqrt{n}}
    \end{equation*}
    Typically we construct $1-\alpha = \SI{95}{\percent}$ confidence intervals.
\end{definition}
\subsection{Hypothesis Testing}
\begin{definition}[Hypothesis Tests]
    In statistics, a hypothesis test allows us to assess how
    consistent a particular assumption (hypothesis) is with our
    observed data.

    The process of hypothesis test starts with an assumption about the
    data. This assumption is known as the ``null hypothesis'', denoted $H_0$.

    We then assess to what extent the sample is inconsistent with the assumption $H_0$,
    by determining how likely we are to observe the sample if the null hypothesis is true.

    This probability is known as the ``$p$-value'' for the test, and indicates how strong the
    evidence is against $H_0$.
    \begin{description}
        \item[p > 0.1] is not sufficient evidence to dispute $H_0$
        \item[0.05 < p < 0.1] might provide weak evidence against $H_0$
        \item[p < 0.05] provides some evidence against $H_0$
        \item[p < 0.01] provides strong evidence against $H_0$
    \end{description}
    The assumption when $H_0$ is rejected is known as the alternative hypothesis, denoted $H_1$ or $H_A$.

    To test the true mean $\mu$, we have
    \begin{equation*}
        H_0:\mu = \mu_0 \quad \text{vs} \quad H_A:\mu \neq \mu_0
    \end{equation*}
    Suppose a sample mean of $\overline{x}$ from a sample of $n$ observations, then:
    \begin{align*}
        \Pr{\left( \abs{\overline{X} - \mu} > \abs{\overline{x} - \mu_0} \right)} & = \Pr{\left( \abs{\frac{\overline{X} - \mu}{s/\sqrt{n}}} > \abs{\frac{\overline{x} - \mu_0}{s/\sqrt{n}}} \right)} \\
                                                                                  & = \Pr{\left( \abs{T} > \abs{\frac{\overline{x} - \mu_0}{s/\sqrt{n}}} \right)}
    \end{align*}
    where $T=\frac{\overline{X} - \mu}{s/\sqrt{n}}\sim t_{n-1}$.
\end{definition}
\section{Linear Regression}
\begin{definition}[Linear Relationship]
    Two variables $x$ and $y$ have a linear relationship if they satisfy an equation of the form
    \begin{equation*}
        y = a x + b
    \end{equation*}
    where $a$ and $b$ are the parameters in the relationship.
\end{definition}
To determine estimates for the values $a$ and $b$, a statistical model must be formulated along with the linear relationship.

To do this, we assume that the points $\left( x_i,\: y_i \right)$ have been observed with some error and formulate the following statistical model
\begin{equation*}
    y_i = \beta_0 + \beta_1 x_i + \varepsilon_i
\end{equation*}
where $\epsilon_i$ is the error in observing $y_i$, and $\beta_0$ and $\beta_1$ are the true values of the parameters.
\begin{definition}[Residual]
    By re-expressing the equation in terms of $\varepsilon_i$,
    \begin{equation*}
        \varepsilon_i = y_i - \beta_0 + \beta_1 x_i
    \end{equation*}
    the residual for each observation can be defined as the amount by which $y_i$ differs from the linear model.
\end{definition}
To make statistical inferences about various features of this model, the following assumptions must be made about the residuals:
\begin{enumerate}
    \item The residuals are independent, identically distributed random variables.
    \item $\forall i : \E{\left( \varepsilon_i \right)} = 0$.
    \item The residuals are normally distributed.
\end{enumerate}
In summary, we assume $\varepsilon_i \sim \mathrm{N}\left( 0,\: \sigma^2 \right)$.
The variance of $\varepsilon_i$ uses a similar form to the expression shown in the previous section.
\begin{align*}
    s^2 & = \frac{1}{n- 2} \sum_i^n \varepsilon_i^2                                          \\
        & = \frac{1}{n- 2} \sum_i^n \left( y_i - \hat{\beta_0} - \hat{\beta_1} x_i \right)^2
\end{align*}
where the hats indicate the estimated values for $\beta_0$ and $\beta_1$.
\subsection{Sample Statistics for \texorpdfstring{$y_i$}{yi}}
These assumptions also allow us to find the properties of $y_i$. As each observation is conducted with a known $x_i$, we can generally assume that it is a constant.
Likewise $\beta_0$ and $\beta_1$ are also constants, therefore the properties of means and variances tells us
\begin{align*}
    \E{\left( y_i \right)}   & = \beta_0 + \beta_1 x_i + \E{\left( \varepsilon_i \right)} = \beta_0 + \beta_1 x_i \\
    \Var{\left( y_i \right)} & = \Var{\left( \varepsilon_i \right)} = \sigma^2
\end{align*}
hence, $y_i \sim \mathrm{N}\left( \beta_0 + \beta_1 x_i,\: \sigma^2 \right)$.
\begin{definition}[Least Squares]
    To estimate $\beta_0$ and $\beta_1$, the intercept and slope of the model,
    we can use the ``least squares criterion'',
    that is, we can minimise the sum of the squares of the residuals.
    \begin{equation*}
        \min_{\beta_0,\: \beta_1}\sum_i^n \varepsilon_i^2 \quad \text{or} \quad \min_{\beta_0,\: \beta_1}\sum_i^n \left( y_i - \beta_0 - \beta_1 x_i \right)^2
    \end{equation*}
\end{definition}
By differentiating this expression w.r.t. $\beta_0$ and $\beta_1$, and setting the derivatives to zero, we reach the following conditions:
\begin{align*}
    \sum_i^n \left( y_i - \beta_0 - \beta_1 x_i \right)    & = 0 \\
    \sum_i^n x_i\left( y_i - \beta_0 - \beta_1 x_i \right) & = 0
\end{align*}
\begin{proof}
    Differentiating w.r.t. $\beta_0$:
    \begin{align*}
        \partial_{\beta_0}\left( \sum_i^n \left( y_i - \beta_0 - \beta_1 x_i \right)^2 \right) & = 0 \\
        \sum_i^n \partial_{\beta_0} \left( y_i - \beta_0 - \beta_1 x_i \right)^2               & = 0 \\
        \sum_i^n 2 \left( y_i - \beta_0 - \beta_1 x_i \right)\left( -1 \right)                 & = 0 \\
        \sum_i^n \left( y_i - \beta_0 - \beta_1 x_i \right)                                    & = 0
    \end{align*}
    Differentiating w.r.t. $\beta_1$:
    \begin{align*}
        \partial_{\beta_1}\left( \sum_i^n \left( y_i - \beta_0 - \beta_1 x_i \right)^2 \right) & = 0 \\
        \sum_i^n \partial_{\beta_1} \left( y_i - \beta_0 - \beta_1 x_i \right)^2               & = 0 \\
        \sum_i^n 2 \left( y_i - \beta_0 - \beta_1 x_i \right)\left( -x_i \right)               & = 0 \\
        \sum_i^n x_i \left( y_i - \beta_0 - \beta_1 x_i \right)                                & = 0
    \end{align*}
\end{proof}
This gives the following estimates for $\beta_0$ and $\beta_1$.
\begin{align*}
    \hat{\beta}_1 & = \frac{\sum_i^n x_iy_i - n\overline{x}\overline{y}}{\sum_i^nx_i^2 - n\overline{x}^2} \\
    \hat{\beta}_0 & = \overline{y} - \hat{\beta}_1\overline{x}
\end{align*}
where $\overline{x} = \frac{1}{n}\sum_i^n x_i$ and $\overline{y} = \frac{1}{n}\sum_i^n y_i$.
\begin{proof}
    For $\hat{\beta}_0$:
    \begin{align*}
        0             & = \sum_i^n \left( y_i - \hat{\beta}_0 - \hat{\beta}_1 x_i \right) \\
        0             & = \sum_i^n y_i - \hat{\beta}_0n - \hat{\beta}_1 \sum_i^n x_i      \\
        \hat{\beta}_0 & = \frac{\sum_i^n y_i - \hat{\beta}_1 \sum_i^n x_i}{n}             \\
        \hat{\beta}_0 & = \overline{y} - \hat{\beta}_1\overline{x}                        \\
    \end{align*}
    For $\hat{\beta}_1$, using $\hat{\beta}_0$:
    \begin{align*}
        0                                                              & = \sum_i^n x_i \left( y_i - \hat{\beta}_0 - \hat{\beta}_1 x_i \right)                                                      \\
        0                                                              & = \sum_i^n x_i y_i - \hat{\beta}_0 \sum_i^n x_i - \hat{\beta}_1 \sum_i^n x_i^2                                             \\
        0                                                              & = \sum_i^n x_i y_i - \left( \overline{y} - \hat{\beta}_1\overline{x} \right) n \overline{x} - \hat{\beta}_1 \sum_i^n x_i^2 \\
        0                                                              & = \sum_i^n x_i y_i - n \overline{x} \overline{y} + \hat{\beta}_1 n \overline{x}^2 - \hat{\beta}_1 \sum_i^n x_i^2           \\
        \hat{\beta}_1 \left( \sum_i^n x_i^2 - n \overline{x}^2 \right) & = \sum_i^n x_i y_i - n \overline{x} \overline{y}                                                                           \\
        \hat{\beta}_1                                                  & = \frac{\sum_i^n x_i y_i - n \overline{x} \overline{y}}{\sum_i^n x_i^2 - n \overline{x}^2}
    \end{align*}
\end{proof}
\subsection{Sample Statistics for \texorpdfstring{$\beta_0$}{beta 0} and \texorpdfstring{$\beta_1$}{beta 1}}
As the estimators $\hat{\beta}_0$ and $\hat{\beta}_1$ are linear combinations of $y_i$, they are also random variables. The variance of these estimates is given by
\begin{align*}
    s_{\hat{\beta}_0}^2 & = s^2 \frac{\sum_i^n x_i^2}{n\sum_i^n \left( x_i - \overline{x} \right)^2} \\
    s_{\hat{\beta}_1}^2 & = s^2 \frac{1}{\sum_i^n \left( x_i - \overline{x} \right)^2}
\end{align*}
where $s^2$ is the estimate for $\Var{\left( y_i \right)}$, (or $\Var{\left( \varepsilon_i \right)}$), as shown previously.
This leads to the following sampling distributions:
\begin{align*}
    \hat{\beta}_0 \sim \mathrm{N}{\left( \beta_0,\: s_{\hat{\beta}_0}^2 \right)} \\
    \hat{\beta}_1 \sim \mathrm{N}{\left( \beta_1,\: s_{\hat{\beta}_1}^2 \right)}
\end{align*}
and it follows that
\begin{equation*}
    \frac{\hat{\beta}_0 - \beta_0}{s_{\hat{\beta}_0}} \sim t_{n-2} \quad \text{and} \quad \frac{\hat{\beta}_1 - \beta_1}{s_{\hat{\beta}_1}} \sim t_{n-2}.
\end{equation*}
\begin{proof}
    To determine the variance for the estimators $\hat{\beta}_0$ and $\hat{\beta}_1$, we must first recognise the following identities:
    \begin{equation}
        \sum_i^n \left( x_i - \overline{x} \right) = 0 \label{eq:sum_of_deviation}
    \end{equation}
    as $\sum_i^n \left( x_i - \overline{x} \right) = \sum_i^n x_i - n \overline{x}$.
    Additionally,
    \begin{equation}
        \Var{\left(\sum_{i}^n a_iX_i\right)} = \sum_i^na_i^2 \Var{\left(X_i\right)} + 2\sum_{i=1}^n\sum_{j=i+1}^na_ia_j\Cov{\left(X_i,\:X_j\right)} \label{eq:variance_of_weighted_sums}
    \end{equation}
    We can now rewrite the numerator and denominator in $\hat{\beta}_1$:
    \begin{align*}
        \sum_i^n x_i y_i - n \overline{x} \overline{y} & = \sum_i^n y_i \left( x_i - \overline{x} \right)                                                                                                                  \\
                                                       & = \sum_i^n y_i \left( x_i - \overline{x} \right) - \sum_i^n \overline{y} \left( x_i - \overline{x} \right) &  & \text{(Using Equation \ref{eq:sum_of_deviation})} \\
                                                       & = \sum_i^n \left( x_i - \overline{x} \right) \left( y_i - \overline{y} \right)
    \end{align*}
    Similarly,
    \begin{align*}
        \sum_i^n x_i^2- n \overline{x}^2 & = \sum_i^n x_i \left( x_i - \overline{x} \right)                                                                                                                  \\
                                         & = \sum_i^n x_i \left( x_i - \overline{x} \right) - \sum_i^n \overline{x} \left( x_i - \overline{x} \right) &  & \text{(Using Equation \ref{eq:sum_of_deviation})} \\
                                         & = \sum_i^n \left( x_i - \overline{x} \right)^2
    \end{align*}
    These two quantities are known as the sum of products and sum of squares, denoted $S_{XY}$ and $S_{XX}$ respectively.

    This allows us to represent $\hat{\beta}_1$ as:
    \begin{equation*}
        \hat{\beta}_1 = \frac{S_{XY}}{S_{XX}} = \frac{\sum_i^n \left( x_i - \overline{x} \right) \left( y_i - \overline{y} \right)}{\sum_i^n \left( x_i - \overline{x} \right)^2}
    \end{equation*}
    To simplify calculations regarding the variance, we will write $S_{XY}$ as $\sum_i^n y_i \left( x_i - \overline{x} \right)$,
    \begin{align*}
        \Var{\left( \hat{\beta}_1 \right)} = s_{\hat{\beta}_1}^2 & = \Var{\left( \frac{\sum_i^n y_i \left( x_i - \overline{x} \right)}{\sum_i^n \left( x_i - \overline{x} \right)^2} \right)}                                                                                                               \\
                                                                 & = \frac{1}{\left( \sum_i^n \left( x_i - \overline{x} \right)^2 \right)^2} \Var{\left( \sum_i^n y_i \left( x_i - \overline{x} \right) \right)}                                                                                            \\
                                                                 & = \frac{1}{\left( \sum_i^n \left( x_i - \overline{x} \right)^2 \right)^2} \Var{\left( \sum_i^n \left( \beta_0 + \beta_1x_i + \varepsilon_i \right) \left( x_i - \overline{x} \right) \right)}                                            \\
                                                                 & = \frac{1}{\left( \sum_i^n \left( x_i - \overline{x} \right)^2 \right)^2} \Var{\left( \sum_i^n \left( \beta_0 + \beta_1x_i \right) \left( x_i - \overline{x} \right) + \sum_i^n \varepsilon_i \left( x_i - \overline{x} \right) \right)} \\
                                                                 & = \frac{1}{\left( \sum_i^n \left( x_i - \overline{x} \right)^2 \right)^2} \Var{\left( \sum_i^n \varepsilon_i \left( x_i - \overline{x} \right) \right)}
    \end{align*}
    Using Equation \ref{eq:variance_of_weighted_sums},
    \begin{align*}
        s_{\hat{\beta}_1}^2 & = \frac{1}{\left( \sum_i^n \left( x_i - \overline{x} \right)^2 \right)^2} \left( \sum_i^n \left( x_i - \overline{x} \right)^2 \Var{\left( \varepsilon_i \right)} + 2\sum_{i=1}^n\sum_{j=i+1}^n\left( x_i - \overline{x} \right)\left( x_j - \overline{x} \right)\Cov{\left(\varepsilon_i,\: \varepsilon_j\right)} \right).
    \end{align*}
    The independence of the residuals implies that the covariance of all residuals -- $\Cov{\left( \varepsilon_i,\: \varepsilon_j \right)}:i \neq j$ is equal to 0. Therefore
    \begin{align*}
        s_{\hat{\beta}_1}^2 & = \frac{1}{\left( \sum_i^n \left( x_i - \overline{x} \right)^2 \right)^2} \sum_i^n \left( x_i - \overline{x} \right)^2 \Var{\left( \varepsilon_i \right)} \\
                            & = \frac{1}{\left( \sum_i^n \left( x_i - \overline{x} \right)^2 \right)^2} \sum_i^n \left( x_i - \overline{x} \right)^2 s^2                                \\
                            & = s^2 \frac{1}{\sum_i^n \left( x_i - \overline{x} \right)^2}
    \end{align*}
    We will use similar techniques to determine the variance of $\hat{\beta}_0$.
    \begin{equation*}
        \Var{\left( \hat{\beta}_0 \right)} = s_{\hat{\beta}_0}^2 = \Var{\left( \overline{y} - \hat{\beta}_1 \overline{x} \right)}
    \end{equation*}
    here we must assume that $\overline{y}$ and $\hat{\beta}_1$ are uncorrelated.
    \begin{align*}
        s_{\hat{\beta}_0}^2 & = \Var{\left( \overline{y} \right)} + \overline{x}^2 \Var{\left( \hat{\beta}_1 \right)}                                                                                   \\
                            & = \Var{\left( \frac{1}{n}\sum_i^ny_i \right)} + \overline{x}^2 s^2 \frac{1}{\sum_i^n \left( x_i - \overline{x} \right)^2}                                                 \\
                            & = \frac{1}{n^2}\Var{\left( \sum_i^ny_i \right)} + s^2 \frac{\overline{x}^2}{\sum_i^n \left( x_i - \overline{x} \right)^2}                                                 \\
                            & = \frac{1}{n^2}\Var{\left( \sum_i^n\left( \beta_0 + \beta_1x_i + \varepsilon_i \right) \right)} + s^2 \frac{\overline{x}^2}{\sum_i^n \left( x_i - \overline{x} \right)^2}
    \end{align*}
    As the residuals are uncorrelated, and $\beta_0$, $\beta_1$, and $x_i$ are all constants, we can simplify the first term further.
    \begingroup
    \allowdisplaybreaks
    \begin{align*}
        s_{\hat{\beta}_0}^2 & = \frac{1}{n^2} \sum_i^n\Var{\left( \beta_0 + \beta_1x_i + \varepsilon_i \right)} + s^2 \frac{\overline{x}^2}{\sum_i^n \left( x_i - \overline{x} \right)^2}                                                        \\
                            & = \frac{1}{n^2} \sum_i^n\Var{\left( \varepsilon_i \right)} + s^2 \frac{\overline{x}^2}{\sum_i^n \left( x_i - \overline{x} \right)^2}                                                                               \\
                            & = \frac{1}{n^2} \sum_i^ns^2 + s^2 \frac{\overline{x}^2}{\sum_i^n \left( x_i - \overline{x} \right)^2}                                                                                                              \\
                            & = \frac{1}{n^2} n s^2 + s^2 \frac{\overline{x}^2}{\sum_i^n \left( x_i - \overline{x} \right)^2}                                                                                                                    \\
                            & = s^2 \frac{1}{n} + s^2 \frac{\overline{x}^2}{\sum_i^n \left( x_i - \overline{x} \right)^2}                                                                                                                        \\
                            & = s^2 \left( \frac{1}{n} + \frac{\overline{x}^2}{\sum_i^n \left( x_i - \overline{x} \right)^2} \right)                                                                                                             \\
                            & = s^2 \frac{\sum_i^n \left( x_i - \overline{x} \right)^2 + n\overline{x}^2}{n \sum_i^n \left( x_i - \overline{x} \right)^2}                                                                                        \\
                            & = s^2 \frac{\sum_i^n \left( x_i - \overline{x} \right)^2 + \sum_i^n \overline{x}^2}{n \sum_i^n \left( x_i - \overline{x} \right)^2}                                                                                \\
                            & = s^2 \frac{\sum_i^n \left( \left( x_i - \overline{x} \right)^2 + \overline{x}^2 \right)}{n \sum_i^n \left( x_i - \overline{x} \right)^2}                                                                          \\
                            & = s^2 \frac{\sum_i^n \left( x_i^2 - 2x_i\overline{x} + \overline{x}^2 + \overline{x}^2 \right)}{n \sum_i^n \left( x_i - \overline{x} \right)^2}                                                                    \\
                            & = s^2 \frac{\sum_i^n x_i^2 - 2\overline{x}\sum_i^n \left( x_i - \overline{x} \right)}{n \sum_i^n \left( x_i - \overline{x} \right)^2}                                                                              \\
                            & = s^2 \frac{\sum_i^n x_i^2}{n \sum_i^n \left( x_i - \overline{x} \right)^2}                                                                                 &  & \text{(Using Equation \ref{eq:sum_of_deviation})}
    \end{align*}
    \endgroup
\end{proof}
\end{document}
\documentclass{article}
\usepackage{template}

\usepackage{chngcntr} % Reset counter within sections
\usepackage{multicol}
\usepackage{textcomp, upquote}

\counterwithin*{equation}{section}
\counterwithin*{equation}{subsection}

\pagestyle{fancy}
\setlength\headheight{24pt}
\setlength\parindent{0pt}

\lhead{\className}
\rhead{\leftmark}
\cfoot{\thepage}

\newcommand{\className}{Engineering Computation}
\newcommand{\classTime}{Semester 2, 2021}
\newcommand{\classInstructorName}{Dr Michael Dallaston}

\usepackage[
    type={CC},
    modifier={by-nc-sa},
    version={4.0},
    imagewidth={5em},
    hyphenation={raggedright}
]{doclicense}

\date{}

\begin{document}

\begin{titlepage}
    \vspace*{\fill}
    \begin{center}
        \LARGE{\textbf{\className}}
        \texorpdfstring{\\}{ }
        \texorpdfstring{\vspace{0.1in}}{ }
        \normalsize{\classTime}
        \texorpdfstring{\\}{ }
        \texorpdfstring{\vspace{0.1in}}{ }
        \normalsize\textit{\classInstructorName}
        \texorpdfstring{\\}{ }
        \texorpdfstring{\vspace{0.2in}}{ }
        \textsc{Tarang Janawalkar}
    \end{center}
    \vspace*{\fill}
    \doclicenseThis
    \thispagestyle{empty}
\end{titlepage}
\newpage

\tableofcontents
\newpage

\lstset{language=Matlab, upquote=true}
\lstset{morekeywords={randi, false, imshow, drawpolygon, polyarea, drawline, ones, imread, VideoWriter, XData, YData, getFrame, writeVideo, deg2rad, soundsc, resample, audiowrite, sind}}
\lstset{
    basicstyle = \ttfamily, columns = fullflexible, keepspaces = true
}

\section{MATLAB Functions}

\begin{table}[H]
    \centering
    \begin{tabular}{c | c}
        \toprule
        Function Syntax & Function Output \\
        \midrule
        \lstinline!y = sin(x)! & Sine with $x$ in radians. \\
        \lstinline!y = sind(x)! & Sine with $x$ in degrees. \\
        \lstinline!y = asin(x)! & Arcsine with $y$ in radians. \\
        \lstinline!y = exp(x)! & $\e^x$. \\
        \lstinline!y = log(x)! & $\ln{\left( x \right)}$. \\
        \bottomrule
    \end{tabular}
    \caption{Common Mathematical Functions in MATLAB.}
\end{table}
All the above functions are element-wise.
\begin{table}[H]
    \centering
    \begin{tabular}{c | c}
        \toprule
        Function Syntax & Function Output(s) \\
        \midrule
        \lstinline!A = zeros(m, n)! & Creates an $m \times n$ matrix containing zeros. \\
        \lstinline!A = ones(m, n)! & Creates an $m \times n$ matrix containing ones. \\
        \lstinline!I = eye(m)! & Creates an $m \times m$ identity matrix. \\
        \lstinline!a = linspace(a, b, x)! & Creates an evenly spaced vector with bounds $\left[a, b\right]$. \\
        \lstinline!y = length(A)! & The largest dimension of $A$. \\
        \lstinline![m, n] = size(A)! & The dimensions of $A$. \\
        \lstinline!y = min(a)! & The minimum value in the vector $a$. \\
        \lstinline!y = max(a)! & The maximum value in the vector $a$. \\
        \bottomrule
    \end{tabular}
    \caption{Matrices and Arrays in MATLAB.}
\end{table}
When manipulating matrices, \verb!*!, \verb!^!, perform matrix operations, while prepending an operator with a dot (\verb!.!) performs an element-wise operation.
\subsection{Plotting}
\begin{table}[H]
    \centering
    \begin{tabular}{c | c}
        \toprule
        Function Syntax & Function Output(s) \\
        \midrule
        \lstinline!plot(x, y)! & Plots given $x$ and $y$ coordinate vectors. \\
        \lstinline!fplot(@f, [a, b])! & Plots the anonymous function over the domain $\left[ a,\: b \right]$. \\
        \lstinline!title('string')! & Adds title to current plot. \\
        \lstinline!xlabel('string')! & Adds $x$-axis label to current plot. \\
        \lstinline!ylabel('string')! & Adds $y$-axis label to current plot. \\
        \lstinline!legend('string1', ...)! & Adds legend to plot. \\
        \lstinline!figure! & Creates a new figure. \\
        \bottomrule
    \end{tabular}
    \caption{Plotting in MATLAB.}
    % \label{}
\end{table}
\section{Operations in MATLAB}
\subsection{Conditional Operations}
\begin{multicols}{2}
    \begin{lstlisting}
if expression
    statements
else if expression
    statements
else
    statements
end
    \end{lstlisting}
    \columnbreak
    Code inside an \lstinline{if} statement only executes if the expression is true. Note that only one branch will execute depending on which expression is true.
\end{multicols}
\subsection{Iterative Operations}
\begin{multicols}{2}
    \begin{lstlisting}
while expression
    statements
end
    \end{lstlisting}
    \columnbreak
    Statements inside a \lstinline{while} loop \linebreak execute repeatedly until the expression is false.
\end{multicols}
\begin{multicols}{2}
    \begin{lstlisting}
for index = values
    statements
end
    \end{lstlisting}
    \columnbreak
    Statements inside a \lstinline{for} loop execute a specific number of times, based on the length of \lstinline{values}.
\end{multicols}
\section{Differential Equations}
\begin{definition}
    A differential equation is an equation that involves the derivatives of a function as well as the function itself.
    An ordinary differential equation (ODE) is a differential equation of a function with only one independent variable.
\end{definition}
\subsection{Electrical Systems}

\section{First-Order Ordinary Differential Equations}
\subsection{Separable ODEs}
\begin{equation*}
    \dv{y}{t} = F(y,\: t) 
\end{equation*}
\begin{enumerate}
    \item Rewrite the equation in the form: $f(y)\dd{y} = g(t)\dd{t}$.
    \item Integrate both sides: $\int f(y)\dd{y} = \int g(t)\dd{t}$.
    \item Rearrange for the explicit form of $y(t)$.
\end{enumerate}
\subsection{Linear ODEs}
Let $P=P(t)$, $Q=Q(t)$ and $\mu = \mu(t)$
\begin{equation*}
    \dv{y}{t} + Py = Q
\end{equation*}
\begin{enumerate}
    \item Determine the integrating factor: $\displaystyle \mu=\exp{\left( \int P \dd{t} \right)}$.
    \item Solve:
    \begin{equation*}
        y=\frac{1}{\mu}\left(\int Q \mu \dd{t} + C\right)
    \end{equation*}
\end{enumerate}
\begin{proof}
    To solve a first-order linear differential equation, determine an integrating factor $\mu = \mu(t)$ such that
    \begin{equation}
        P \mu = \dv{\mu}{t} \label{eq:integrating_factor}
    \end{equation}
    Multiplying the equation by $\mu$ gives
    \begin{align*}
        \mu \dv{y}{t} + P \mu y &= Q \mu \\
        \mu \dv{y}{t} + \dv{\mu}{t} y &= Q \mu \\
        \dv{t}\bigl(\mu y\bigr) &= Q \mu \\
        \int \dv{t}\bigl(\mu y\bigr) \dd{t} &= \int Q \mu \dd{t} \\
        \mu y &= \int Q \mu \dd{t} \\
        y &= \frac{1}{\mu}\left(\int Q \mu \dd{t} + C\right)
    \end{align*}
    To determine $\mu$ we can rearrange \hyperref[eq:integrating_factor]{Equation \ref{eq:integrating_factor}} into
    \begin{align*}
        P = \frac{1}{\mu}\dv{\mu}{t}
    \end{align*}
    By recognition, this is the derivative of the natural logarithm of $\mu$ with respect to $t$.
    \begin{align*}
        P &= \dv{t}(\ln{\left( \mu \right)}) \\
        \int P \dd{t} &= \int \dv{t}\bigl(\ln{\left( \mu \right)}\bigr) \dd{t} \\
        \int P \dd{t} &= \ln{\left( \mu \right)} \\
        \mu &= \exp{\left( \int P \dd{t} \right)}
    \end{align*}
\end{proof}
\subsection{Solution using Linearisation}
A function can be linearised by using its 1st degree Taylor polynomial near $a$.
\begin{equation*}
    f(x) \approx f(a) + f'(a)(x-a) + \mathcal{O}(x^2)
\end{equation*}
This new polynomial can be substituted to form a linear ODE, which can be solved using an integrating factor.
\newpage
\section{Second-Order Ordinary Differential Equations}
\subsection{Constant Coefficient Linear ODEs}
\begin{equation*}
    a \dv[2]{y}{t} + b \dv{y}{t} + c y = Q(t)
\end{equation*}
where $a,\:b,\:c$ are constants.
\subsection{Linearity of Solutions}
\begin{theorem}[Principle of Superposition]\label{theorem:superposition}
    As the given ODE is linear, if $y_1(t)$ is a solution to the equation
    \begin{equation*}
        a \dv[2]{y_1}{t} + b \dv{y_1}{t} + cy_1 = Q_1(t)
    \end{equation*}
    and $y_2(t)$ is a solution to 
    \begin{equation*}
        a \dv[2]{y_2}{t} + b \dv{y_2}{t} + cy_2 = Q_2(t)
    \end{equation*}
    then for the function $y = c_1y_1+c_2y_2$
    \begin{equation*}
        a \dv[2]{y}{t} + b \dv{y}{t} + cy = c_1Q_1(t) + c_2Q_2(t)
    \end{equation*}
    where $c_1$ and $c_2$ are constants.
\end{theorem}
% \begin{proof}
% \begin{align*}
%     a \dv[2]{y}{t} + b \dv{y}{t} + cy &= a \dv[2]{t}(k_1y_1+k_2y_2) + b \dv{t}(k_1y_1+k_2y_2) + c(k_1y_1+k_2y_2) \\
%     &= k_1\left( a \dv[2]{y}{t} + b \dv{y}{t} + cy \right)a \dv[2]{t}(k_1y_1+k_2y_2) + b \dv{t}(k_1y_1+k_2y_2) + c(k_1y_1+k_2y_2) \\
% \end{align*}
% \end{proof}
\subsection{Homogeneous ODEs}
\begin{definition}
    A homogeneous ODE has $Q(t)=0$, which gives
    \begin{equation*}
        a \dv[2]{y}{t} + b \dv{y}{t} + c y = 0
    \end{equation*} 
\end{definition}
This differential equation has a solution of the form:
\begin{equation*}
    y_h = \e^{rt}
\end{equation*}
\subsection{Characteristic Equation}
By making the substitution $y=\e^{rt}$, we get
\begin{align*}
    a \dv[2]{y_h}{t} + b \dv{y_h}{t} + c y_h &= 0 \\
    ar^2\e^{rt} + br\e^{rt} + c\e^{rt} &= 0 \\
    (ar^2 + br + c)\e^{rt} &= 0 \\
    ar^2 + br + c &= 0
\end{align*}
This is known as the characteristic or \textit{auxiliary} equation. The next step is to calculate the roots of the equation.
\begin{description}
    \item[Real Distinct Roots.] If $b^2 > 4ac$.
    \item[Real Repeated Roots.] If $b^2 = 4ac$.
    \item[Complex Conjugate Roots.] If $b^2 < 4ac$.
\end{description}
\subsubsection{Real Distinct Roots}
Given $r_1$ and $r_2$ are real and distinct:
\begin{align*}
    y_1(t) = \e^{r_1t} & & y_2(t) = \e^{r_2t}
\end{align*}
Hence the solution to the homogeneous equation is given by:
\begin{equation*}
    y_h(t) = c_1\e^{r_1t} + c_2\e^{r_2t}
\end{equation*}
\subsubsection{Real Repeated Roots}
Given $r$ is a repeated root:
\begin{align*}
    y_1(t) = \e^{rt} & & y_2(t) = t\e^{rt}
\end{align*}
Hence the solution to the homogeneous equation is given by:
\begin{equation*}
    y_h(t) = c_1\e^{rt} + c_2t\e^{rt}
\end{equation*}
\subsubsection{Complex Conjugate Roots}
Given $r_1 = \alpha + \beta i$ and $r_2 = \alpha - \beta i$ are complex conjugates:
\begin{align*}
    y_1(t) = \e^{r_1t} & & y_2(t) = \e^{r_2t}
\end{align*}
Hence the solution to the homogeneous equation is given by:
\begin{equation*}
    y_h(t) = c_1\e^{\alpha t}\cos{\left( \beta t \right)} + c_2\e^{\alpha t}\sin{\left( \beta t \right)}
\end{equation*}
\subsection{Nonhomogeneous ODE}
A nonhomogeneous differential equation is of the form
\begin{equation*}
    a \dv[2]{y}{t} + b \dv{y}{t} + c y = Q(t)
\end{equation*}
where $Q(t)\neq 0$.
\subsection{General Solution of a Nonhomogeneous ODE}
Recall that the solutions to any linear ODE are additive, so that if a solution $y_p$ satisfies 
the nonhomogeneous ODE, and $y_h$ satisfies the homogeneous ODE,
\begin{equation*}
    y = y_h + y_p
\end{equation*} 
must also satisfy the ODE.
\subsection{Undetermined Coefficients}
To solve for $y_p$, we substitute a guess like the homogeneous case, and
the coefficients in this guess will be determined from the ODE itself.

The particular solution will depend on what $Q(t)$ looks like.
\begin{table}[H]
    \centering
    \begin{tabular}{c | c}
        \toprule
        $Q(t)$ & $y_p$ \\
        \midrule
        a constant & $A$ \\ 
        $n$th degree polynomial & $A_0 + A_1t + \cdots + A_{n-1}t^{n-1} + A_nt^n$ \\ 
        $\e^{\alpha t}$ & $A\e^{\alpha t}$ \\ 
        $\cos{\left( \alpha t \right)}$ & $A\cos{\left( \alpha t \right)} + B\sin{\left( \alpha t \right)}$ \\ 
        $\sin{\left( \alpha t \right)}$ & $A\cos{\left( \alpha t \right)} + B\sin{\left( \alpha t \right)}$ \\ 
        $\cos{\left( \alpha t \right)} + \sin{\left( \alpha t \right)}$ & $A\cos{\left( \alpha t \right)} + B\sin{\left( \alpha t \right)}$ \\ 
        \bottomrule
    \end{tabular}
    \caption{Particular Solutions for Undetermined Coefficients}
    % \label{}
\end{table}
\subsection{Special Forms}
\subsubsection{Product of Forms}
If $Q(t)$ is a product of the functions shown above, then we write the 
particular solution for both functions separately and multiply the results together.

For example, with $Q(t) = t \e^{4t}$, we have
\begin{equation*}
    y_p = \left(At + B\right) \left(C\e^{4t}\right)
\end{equation*}
the next step is to expand the function simplify any constants.
\begin{align*}
    y_p &= \left(ACt + BC\right) \e^{4t}
    y_p &= \left(A_1t + B_1\right) \e^{4t}
\end{align*}
\subsubsection{Sum of Forms}
If $Q(t)$ is a sum of the functions shown above, then we can use 
\hyperref[theorem:superposition]{Theorem \ref{theorem:superposition}}
and add the particular solutions together.
\subsubsection{Linearly Dependent Case}
If $Q(t)$ is similar to any homogenous solution, then by \textit{definition} of 
a homogeneous solution, the solution will be $0$. Hence, $y_p$ must be multiplied by $t$ 
to ensure that the particular solution is linearly independent to the homogeneous solutions,
in order to form a \textit{fundamental set of solutions}. 
\subsection{Solving the Particular Solution}
\begin{enumerate}
    \item Solve $y_h$
    \item Find an appropriate form for $y_p$
    \item Ensure that $y_p$ is linearly independent to the homogeneous solutions
    \item Substitute $y_p$ into the nonhomogeneous ODE and solve for the undetermined coefficients
    \item Find the general solution $y = y_h + y_p$
    \item Apply initial conditions to solve for any constants
\end{enumerate}
\section{Systems of Ordinary Differential Equations}
A first-order system of differential equations has the form
\begin{equation*}
    \left\{
        \setlength\arraycolsep{0pt}
        \begin{array}{ c >{{}}c<{{}} c >{{}}c<{{}} c >{{}}c<{{}} c >{{}}c<{{}} c  }
        x'_1               &=& a_{11}x_1                         &+& a_{12}x_2                         &+& \cdots &+& a_{1n}x_n \\
        x'_2               &=& a_{21}x_1                         &+& a_{22}x_2                         &+& \cdots &+& a_{2n}x_n \\
        \vdotswithin{x'_3} & & \vdotswithin{a_{31}}\phantom{x_1} & & \vdotswithin{a_{32}}\phantom{x_2} & &        & & \vdotswithin{a_{3n}}\phantom{x_n} \\ 
        x'_n               &=& a_{n1}x_1                         &+& a_{n2}x_2                         &+& \cdots &+& a_{nn}x_n 
        \end{array}
    \right.
\end{equation*}
where $x_1=\func{x_1}{t},\: x_2=\func{x_2}{t},\: \dots,\: x_n=\func{x_n}{t}$ are the 
functions to be determined. In matrix form, the system can be written as
\begin{align*}
    \dv{t}\mqty[x_1 \\ x_2 \\ \vdots \\ x_n] &= \mqty[
        a_{11} & a_{12} & \cdots & a_{1n} \\
        a_{21} & a_{22} & \cdots & a_{2n} \\
        \vdots & \vdots &        & \vdots \\
        a_{n1} & a_{n2} & \cdots & a_{nn}
    ] \mqty[x_1 \\ x_2 \\ \vdots \\ x_n] \\
    \symbf{x}' &= \symbfit{A} \symbf{x}
\end{align*}
\subsection{Higher-Order ODEs}
A higher-order linear differential equation can be solved by first converting it to a first-order linear 
system. Consider the $n$th-order homogeneous differential equation
\begin{equation*}
    y^{\left( n \right)} + a_1 y^{\left( n-1 \right)} + \cdots + a_{n-1} y' + a_n y = 0
\end{equation*}
Let
\begin{align*}
    x_1 &= y \\
    x_2 &= y' \\
    &\vdotswithin{=} \\
    x_n &= y^{\left( n-1 \right)}
\end{align*}
so that $\symbfit{x}=\mqty[x_1 & x_2 & \cdots & x_n]^\top$. Then the differential equation can
be expressed as the following first-order linear system of differential equations
\begin{equation*}
    \dv{t}\mqty[
        x_1 \\
        x_2 \\
        \vdotswithin{x_3} \\
        x_n	
    ] = \mqty[
        0 & 1 & 0 & \cdots & 0 \\
        0 & 0 & 1 & \cdots & 0 \\
        \vdots & \vdots & \vdots & \ddots & \vdots \\
        0 & 0 & 0 & \cdots & 1 \\
        -a_n & -a_{n-1} & -a_{n-2} & \cdots & -a_1
    ] \mqty[
        x_1 \\
        x_2 \\
        \vdotswithin{x_3} \\
        x_n	
    ]
\end{equation*}
\subsection{Solution Form}
Like the homogeneous case, we will guess a solution of the form
\begin{equation*}
    \symbf{x} = \symbf{q}\e^{\lambda t}
\end{equation*}
which allows for the following substitution
\begin{align*}
    \lambda \symbf{q}\e^{\lambda t} &= \symbfit{A} \symbf{q}\e^{\lambda t} \\
    \symbfit{A} \symbf{q}\e^{\lambda t} - \lambda \symbf{q}\e^{\lambda t} &= \symbf{0} \\
    \left( \symbfit{A} - \lambda\symbfit{I} \right) \symbf{q}\e^{\lambda t} &= \symbf{0} \\
    \left( \symbfit{A} - \lambda\symbfit{I} \right) \symbf{q} &= \symbf{0}
\end{align*}
This equation has the trivial solution $\symbf{q}=\symbf{0}$, however for a fundamental set of solutions,
we must let $\symbfit{A} - \lambda\symbfit{I}$ be singular.
\subsubsection{Characteristic Equation}
To determine the eigenvalues $\lambda$ of the matrix $\symbf{A}$, we must solve the characteristic equation
associated with the system of ODEs. Namely,
\begin{equation*}
    \det{\left( \symbfit{A} - \lambda\symbfit{I} \right)} = 0
\end{equation*}
These eigenvalues can then be used to solve the eigenvectors of $\symbfit{A}$
\subsection{Solving a System of ODEs}
\begin{enumerate}
    \item Model the system of ODEs in the form $\symbf{x}' = \symbfit{A} \symbf{x}$
    \item Solve the characteristic equation for the eigenvalues of $\symbfit{A}$
    \item Solve the corresponding eigenvectors of $\symbfit{A}$ by solving $\left( \symbfit{A} - \lambda\symbfit{I} \right) \symbf{q} = \symbf{0}$
    \item Write the general solution: $\symbf{x} = c_1\symbf{q}_1\e^{\lambda_1 t} + c_2\symbf{q}_2\e^{\lambda_2 t}$
    \item Apply initial conditions to solve $c_1$ and $c_2$
\end{enumerate}
\section{Probability}
\begin{definition}[Random Variables]
    A random variable $X$ is a measurable variable that doesn't hold a definitive value.
\end{definition}
\begin{definition}[Discrete Random Variables]
    A discrete random variable $X$ has a countable number of possible values.
\end{definition}
\begin{definition}[Continuous Random Variables]
    A continuous random variable $X$ can take all values in a given interval.
\end{definition}
\begin{definition}[Probability]
    Probability is used to mean the chance that a particular event 
    (or set of events) will occur, expressed on a linear scale from 0 to 1.
    The probability of the random variable $X$ taking the value $x$ is denoted
    \begin{equation*} 
        \Pr{\left( X = x \right)}   
    \end{equation*} 
\end{definition}
\begin{definition}[Sample Space]
    Let the set of all possible outcomes of a random variables be called the sample
    space, denoted $\Omega$, of that random variable. 
    
    Let $X$ be a random variable that can take on values $x\in\Omega$. Then
    for all $x\in\Omega$ there is an associated probability $p(x)$, such that
    \begin{equation*}
        \forall x\in\Omega : 0 < p(x) \leq 1
    \end{equation*}
    \begin{equation*}
        \sum p(x) = 1.
    \end{equation*}
\end{definition}
\subsection{Events}
\begin{definition}[Events]
    An event $A$ is a set of individual outcomes within $\Omega$. 
    Then for some event $A\subset\Omega$, the probability is given by
    \begin{equation*}
        \Pr{\left( A \right)} = \sum_{x\in A} p(x).
    \end{equation*}
    The complementary event, denoted $A^C$ (also $\overline{A}$) is the set of all 
    outcomes within the sample space that are not within $A$.
    \begin{equation*}
        \Pr{\left( A^C \right)} = 1 - \Pr{\left( A \right)}.
    \end{equation*}
\end{definition}
\begin{theorem}[Combination of Events]
    Events can be combined with the two logical connectors AND and OR, which are
    equivalent to the intersection ($\cap$) and union ($\cup$) of set.
\end{theorem}
\begin{theorem}[Mutually Exclusive Events]
    If two events have no possible outcomes in common, they are mutually exclusive or disjoint events.
    \begin{equation*}
        A \cap B = \varnothing.
    \end{equation*}
    It follows that
    \begin{equation*}
        \Pr{\left( A \cap B \right)} = 0.
    \end{equation*}
\end{theorem}
\begin{theorem}[Probability of Union]
    \begin{equation*}
        \Pr{\left( A \cup B \right)} = \Pr{\left( A \right)} + \Pr{\left( B \right)} - \Pr{\left( A \cap B \right)}
    \end{equation*}
\end{theorem}
\begin{theorem}[AND Statements]
    
\end{theorem}
\subsection{Dependent Events}
\begin{definition}[Conditional Probability]
    
\end{definition}
\begin{theorem}[Total Probability]
    
\end{theorem}
\begin{theorem}[Bayes' Rule]
    
\end{theorem}
\section{Probability Distributions}
\begin{definition}[Expectation]
    
\end{definition}
\begin{definition}[Variance]
    
\end{definition}
\subsection{Binomial Distribution}
\subsection{Bernouilli Distribution}
\subsection{Poisson Distribution}
\subsection{Uniform Distribution}
\subsection{Exponential Distribution}
\subsection{Normal Distribution}
\end{document}